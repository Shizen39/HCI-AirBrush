% !TeX spellcheck = IT
\documentclass[a4paper, twoside]{article}
\usepackage[utf8]{inputenc}
\usepackage[italian]{babel}
\usepackage{graphicx}
\usepackage{float} %per posizione assoluta delle immagini [H]
\usepackage[colorlinks=true,linkcolor=blue]{hyperref}
\usepackage{nameref} 
\usepackage{siunitx} 
\usepackage[top=0.7in, bottom=0.7in, left=0.7in, right=1.6in]{geometry} %margini
\setcounter{section}{-1} %per numerare le sezioni partendo da 0

\graphicspath{{./images/}}

%defines-----------------------------------
\def\code#1{\texttt{#1}}
\def\sec#1{\section{#1}\label{#1}} 
\def\sub#1{\subsection{#1}\label{#1}} 
\def\subsub#1{\subsubsection{#1}\label{#1}} 
\def\para#1{\paragraph{#1}\label{#1}}
\def\vedi#1{\nameref{#1}} 
\def\italic#1{\textit{#1}}
\def\image[#1][#2]#3{
  \begin{figure}[H]
  \centering
  \includegraphics[#2]{#1}
  \caption{#3}
  \end{figure}}
\def\boximage[#1][#2]#3{
  \begin{figure}[H]
  \centering
  \fbox{\includegraphics[#2]{#1}}
  \caption{#3}
  \end{figure}}
\def\footurl[#1]#2{\href{#2}{#1}\footnote{\url{#2}}}
%------------------------------------------

\title{AirBrush}
\author{Giorgio Mazza}

\begin{document}

\maketitle
\newpage
\tableofcontents
\newpage

\section{Generalità}
\subsection{Descrizione del prodotto}
In ambiti artistici quali quelli del disegno o della modellazione di oggetti quali vestiti, mobili, modellini o modelli usabili nei videogiochi, è sempre stato presente un grande vincolo: modellare oggetti intrinsecamente tridimensionali, in uno spazio bidimensionale (il monitor).\\
AirBrush è un'applicazione per smartphones e tablet che permette agli utenti di approcciarsi al disegno e alla modellazione 3d in modo del tutto innovativo. \\
Grazie alle nuove tecnologie, AirBrush introduce la Realtà Aumentata (AR) nei suddetti ambiti; Ciò le permette di rompere i vincoli imposti dallo schermo del monitor e di immergere l'utente in un'esperienza creativa intuitiva, innovativa, efficace e totalmente libera.\\
Così lo spazio di lavoro, la tela su cui dipingere, lo spazio in cui modellare, diventano lo spazio che circonda l'utente.

\subsection{Obiettivi}
\paragraph{Disegno/Modellazione}: L'app intende rivoluzionare l'approccio dei designers (con esperienza o meno), ma anche di utenti appassionati o incuriositi, al disegno e alla modellazione 3d. \\
L'approccio bidimensionale a cui siamo abituati può sembrare a volte complesso quanto poco efficace. Per non parlare poi della curva di apprendimemnto di determinati software specifici. E' qui che entra in gioco AirBrush rompendo gli schemi a cui siamo abituati, quali l'uso del monitor, di un foglio o di una tela per dare sfogo alla nostra creatività, facendo salire l'esperienza utente ad un livello superiore e totalmente innovativo.
\paragraph{Market}: L'app si  pone anche come servizio di condivisione dei propri modelli o disegni con l'intera community, istituendo una classifica dei migliori lavori con annesso sistema di ranking, di conteggio delle visualizzazioni e dei downloads, divenendo di fatto per l'artista occasione di visibilità e di guadagno attraverso i propri lavori.\\

\subsection{Utenti}
\subsection{Scenari d'uso}
\subsection{Fattibilità tecnologica}

\section{Posizionamento}
\subsection{Situazione attuale}
\subsection{Posizionamento competitivo}
\subsection{Swot Analysis}

\section{Requisiti}
\subsection{Casi d'uso}
\subsection{Requisiti per l'esperienza utente}

\section{TODO}

0) onboarding (modellazione free + market in lettura) -$>$ 1) navigazione free -$>$  [esportazione in locale/pubblicazione market] -$>$  2) accesso -$>$ 3) uso\\\\


0.Modellazione free\\
-Strumenti modellazione base		--$>$ 	1.Modellazione con registrazione\\				
-Accesso in sola lettura al market	--$>$ 	2.Market con registrazione\\\\

1.Modellazione con registrazione\\
-Accesso a strumenti avanzati \\
-Possibilità di importare modelli\\\\

2.Market\\
-Guadagnare pubblicando i tuoi lavori + acquisto modelli \\
-Views + downloads  -$>$  classifica  -$>$  visibilità\\
-Categorie modelli + tag per aumentare visibilità  -$>$  profilazione\\

\end{document}